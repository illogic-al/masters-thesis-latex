\chapter{Technical Report Writing}

A thesis is a technical report.  Read and follow a good style book
(e.g., Chicago) on technical report writing.  It is a good exercise to
mark {\em this} article up!  Additionally, every university has
guidelines regarding layouts and required pages such as a title page
an approval page.

(This article is itself poorly and hastily written.  Think of it as a
collection of notes.  ``Do as I say, not as I do! {\tt ;-)}'' )


\section{Structure}

When you are moving from one depth to a deeper one there is always an
introductory paragraph.  E.g., going from a chapter heading to its
first section, and from a section heading to its first subsection.

Sometimes we like to write a paragraph at the end of a
chapter that does not belong to the last section.  This is often done
by introducing a vertical white space after the end of the last
section that is longer than the typicial spacing between consecutive
sections.

There are no particular size requirements.  But, a one page chapter
and a two line section do look silly.

\section{URLs}

URLs are now common in CS theses.  Older style guides do not cover these.
Include a URL inside \verb|\url{}|. Example:
\url{http://www.google.com/search?q=latex+eepic+pstricks+tikz}
\section{Acronyms}

Acronyms are written in all-upper case.  Examples: GNU, BCPL, KDE.

Linux, Unix, Pascal, Ada, Java, ... are not acronyms.  LINUX -> Linux,
no boldface, L in caps, the rest in lower case.

\section{References}


The file named {\tt thesis.bib} should contain the references you collected
in the {BibTeX} syntax.  I have included a few references to
illustrate this syntax; you can leave them in this file.  In general,
it can contain many more references than what you cite in the body of
your thesis.  The {\tt bibtex} program examines the actual citations made,
and pulls together the cited references into the text file named
{\tt thesis.bbl}.


Bibliographic citations are done in many ways.  I require my students to
use the {\tt acmtrans.bst} style.  Other acceptable choices are those
that include author(s) last name and year of publication in the body
of the text that cites, and the end of the article list of references
are sorted alphabetically by the first author's last name.  Styles
that produce a bracketed number, as in [5], are not acceptable to me.

When using \url{http://scholar.google.com/}, set Bibliography Manager
to use BibTeX in the preferences.  Then, the references found
can be saved in BiBTeX{} format.  Visit also
\url{http://www.bibsonomy.org/}.

% -eof-
