\chapter{Discussion}

As demonstrated in figures~\ref{fig:orai3_digests} and \ref{fig:stim1_digests} the ligation of the inserts and subsequent cloning of \oraiiiivector{} and \stimivector{} were successful. \droso{} vectors \emph{capable} of expressing mammalian Orai3 and \stim{} genes in S2 cells were created, satisfying specific aim \#1. 

Figures~\ref{fig:orai3_rtpcr} and \ref{fig:stim1_rtpcr} then showed induction of Orai3 and \stim{} \emph{gene expression} from these vectors. This demonstrated that both vector constructs actually expressed these mammalian genes in \droso{} S2 cells, satisfying specific aim \#2. Having carried out the objectives in specific aims \#1 and \#2, we moved on to address specific aim \#3: assessing the effect of 2-APB on Orai3 channels. As mentioned previously, concentrations of $\ge$ 50 $\mu$M 2-APB will activate mammalian Orai3 channels.

Initial attempts at taking \Ca{} measurements in S2 cells were thwarted by leakage of Fura-2, the \Ca{} dye being used, from these cells. As demonstrated in figures~\ref{fig:s2_no_pro}, \ref{fig:s2_with_pro} and \ref{fig:s2_pro_compare} addition of the anion transport inhibitor probenecid was crucial to recording satisfactory calcium transients. We see that omission of probenecid led to ratio measurements of SOCE transients which were lower than those recorded when probenecid was present.  Leakage of Fura-2, would have resulted in a gradual elevation of the significance of the background fluorescence. The lower ratio maxima for SOCE transients then is likely due to the low ratio caused by background fluorescence in the the cells not treated with probenecid.
%the result of less Fura-2 being present for binding to \Ca{} due to a persistent leak. 
We see that addition of probenecid reduced dye leakage by blocking the transporters responsible, and resulted in better measurements of S2 cell populations. This demonstrates that the use of probenecid for \Ca{} recordings of \droso{} S2 cells is necessary for obtaining quality data \citep{Yagodin1999}. 
 
After solving the issue of recording \Ca{} transients, assessing the effect of 2-APB on S2 cells expressing Orai3 was possible. Our hypothesis, that gene expression of mammalian Orai3 in \droso{} S2s would result in \Ca{} channels that behaved similarly to those in mammalian cells expressing Orai3 proved incorrect. As seen by the results presented in figures~\ref{fig:s2_composite} and \ref{fig:s2_2apb_bar}, Orai3 channel behavior in our  \droso{} expression system was anomalous, no increase in SOCE due to Orai3 was observed (2 Ca column, figure \ref{fig:s2_2apb_bar}). Aims \#1 and 2 were achieved, but testing of aim \#3 showed that the expression model requires further optimization. 

An unexpected and interesting result from this study was the ability of \stim{} to cause an increase in cytosolic \Ca{}. The results in figure~\ref{fig:s2_composite} for the \stim{} only transfection hint at less \emph{deactivation} of \dorai{} occurring in 50 $\mu$M 2-APB  compared to the mock transfection. Addition of 200 $\mu$M 2-APB to \stim{} only transfected S2s resulted in a clear increase in cytosolic calcium indicative of channel opening.  
One explanation is that another \Ca{} channel opened due to the combination of \stim{} and 2-APB. It may also be that 200 $\mu$M 2-APB is interacting with \stim{} leading to non-specific effects on \dorai. 
As this is only present in the \stim{} transfected cells, and since the amount of \stim{} DNA used in \stim{} and Orai3+\stim{} transfections was the same (2 $\mu$g) this suggests an effector+2-APB interaction in the absence of Orai3. 

%The presence of mammalian \stim{} with 200 µM 2-APB, while not able to increase SOCE, slows the inactivation of \dorai. This results in a significant increase of intracellular calcium, when compared to the mock and other transfections. Support for theory of alteration of pore selectivity? If the slowing of inactivation of \dorai{} currents was due to an increase in the 2-APB which somehow affected the SERCA pump, a drug whose effects are not reversible (personal communication, Dr. J. Ashot Kozak), then we would expect to see similar results in all the other transfections, due to the effects of the hypothetical CPA washout. 
%Figure~\ref{fig:s2_2apb_bar}

%Discussion mock:
%To determine whether the cell's cytosol was at its maximal \Ca{} carrying capacity, experiments with 2 Ca and ionomycin can be performed. After store depletion with CPA, 2 Ca can be perfusion, followed by a 2 Ca + ionomycin solution. Ionomycin is an ionophore which would raised the   \cai{} to that of the external solution. Depending on the result of perfusion with 2 Ca + ionomycin, we could say with certainty whether the maximal \Ca{} carrying capacity of the cytoplasm was reached. If for example, perfusion with ionomycin resulted in higher Ca{} levels, than was obtained without ionomycin, this would indicate that maximal levels were not obtained. 

%Another method for determining this would be to change the concentration of \Ca{} being used in the perfusion solution. All solutions contained 2 mM \CaCl, but decreasing the concentration to $\mu$M levels, which provide less \Ca{} for SOCE. This would result in less Ca{} being available for entry and a lower cytosolic calcium content resulting. If the \Ca{} from SOCE was still not significantly different between the transfection groups, this would allow us to conclude that the Orai3 channels were not activated solely as a function of store-depletion.



%By using directional cloning techniques we were able to quickly generate our desired constructs. Restriction digest analysis was used to determine whether Orai3 and \stim{} were inserted into \puchygmt{} in the intended orientation. The size of the \sali{}+\xbai{} digested vector was taken to be quite close to 7000 bp, as no clear band was visible at 100 bp or higher. Given that these sites were in the multiple cloning region, and given the closeness of the sites on the provided map, it is likely that only a few base pairs separated the sites. The original vector map can be seen in the supplemental data.

%In figure~\ref{fig:orai3_digests} \bamhi{} and \xhoi{} were used to digest \oraiiiivector{} and determine the insert size. In the absence of an insert, a sole $\sim$7000 bp would be observable on the gel. The presence of the $\sim$6957 bp and 931 bp bands indicated that Orai3 insertion was successful.

%\bamhi{} and \xhoi{} were selected because \sali{} and \xbai{} digestion did not produce the expected bands. Further analysis uncovered that this was due to the \xbai{} site being blocked as a result of a dam methylation. The {\tt TGA} stop codon of Orai3 combined with the {\tt TCTAGA} \xbai{} restriction site, introduced a {\tt GATC} dam methylation site. The result was a linearized \oraiiiivector{} after \sali+\xbai{} double digestion, because \xbai{} is blocked by dam methylation.
%See supplemental figure.\todo{create supplemental figure for this}are present in \puchygmt{}. 

%The use of \kpni{}, whose site is specific to Orai3 and \xhoi, specific to the vector allowed us to determine whether Orai3 inserted properly. Successful insertion in the 5\'{} to 3\'{} direction, would yield the 796 bp band observed in lane 3 of figure~\ref{fig:orai3_digests}. 
%After restriction digest analysis, clones which displayed the expected bands were sent for sequence verification. 
%Sequence analysis on \oraiiiivector{} found no mutations in the Orai3 sequence. This experiment and the subsequent sequence analysis were important in providing confirmation that the \oraiiiivector{} construct was successfully generated. %The sequence of \oraiiiivector, can be seen in the supplemental data.


%With our \stimivector{} construct, there were no issues with dam methylation, and so \xbai{} could be used. Lane 4 shows the 2066 bp and $\sim$6994 bp bands we expect from \sali+\xbai{} digestion, along with a linearized $\sim$9060 bp product. The result of \stui+\xhoi{} and \bamhi+\stui{}  digestions in lanes 1 and 2 respectively,  confirm insertion of \stim in the correct direction. Together, these restriction digests confirm that the \stimivector{} was also successfully created, and in the proper direction. Following restriction digest analysis, \stimivector{} clones which displayed the expected bands were also sent for sequence verification. Sequence analysis on \stimivector{} DNA confirmed the sequence and direction of \stim{} insertion as well.  



There are several possibilities for why Orai3, either alone or with \stim{}, did not display the expected \Ca{} increase after addition of 50 or 200 $\mu$M 2-APB. The most obvious is that while Orai3 RNA production was induced with \cuso, actual protein production did not occur by the time of the experiments. Another possibility is that the transfection efficiency was consistently low with the transfection reagent and \oraiiiivector{} DNA. This may have resulted in poor expression of Orai3 and a paucity of Orai3 protein translation, leading to a deficiency in Orai3 channels expressed at the cellular surface.
Another possibility is that, because of how closely related Orai3 and \dorai{} are -- a BLAST comparison shows a 64\% nucleotide sequence similarity -- Orai3 forms heteromers with \dorai, and adopts a phenotype primarily \dorai{} dominated in nature.  

Given that \stim{} expression results in an unexpected phenotype, it is unlikely that Orai3, the smaller protein which was made in a similar manner, does not express. It is more likely that interactions with the closely related \dorai{} result in heteromeric channels, and it is the output from those channels that we are investigating.

Transfection of Orai3 with \stim{} did lead to a significant increase in cytosolic \Ca{} compared to Orai3 only transfected cells (see figure~\ref{fig:s2_2apb_bar}). This suggests that \dstim{} is not sufficient for Orai3 function.
 
Future work on this project will involve testing the possibilities presented above, for why Orai3 activation did not occur after 2-APB treatment. The generation of stable cell lines using the hygromycin B resistance conferred by \puchygmt{}  is underway. While creating clonal populations of stably selected S2 cells is possible, reports indicate that the additional effort does not seem worthwhile \cite{Schetz2004}. Expression levels for high-expressing clones seem similar to those from a polyclonal population \cite{Schetz2004}.
Stable lines expressing  Orai3 and \stim{} together will facilitate experiments needed to test these possibilities. 

Protein expression of Orai3 can be tested by isolating proteins from transfected cells and performing Western Blot analysis using antibodies to Orai3. Testing for interference from native \dorai{} by performing \rnai{} knockdown will address the issue of \dorai{} influencing the Orai3 channel phenotypes. Beyond the potential issue of \dorai{} affecting Orai3 function and/or expression, \rnai{} knockdown of \dorai{} allows study of Orai3 channels in isolation on a \dorai{}-free background. This will be useful for defining Orai3 channel function, as opposed to general SOCE, to which Orai3 may contribute.

\section{Conclusion}
The creation of an expression system for mammalian Orai \Ca{} channels to be used for drug discovery purposes was begun in this study. The proteins selected to probe the feasibility of using such a system were the mammalian Orai3 \Ca{} channel, and mammalian \Ca{} sensor \stim. Constructs for expression in S2 cells were created and were able to successfully express genes from both Orai3, and \stim. Our \droso{} system was not able to reproduce the function of the Orai3 channel in the presence of the known effector 2-APB, highlighting the need for further refinement, before use in future drug studies. Provided that the issues with channel expression are solved, the system has the potential to be a very useful tool, for study of not just Orai3, but the entire Orai family. 

The system is effective in inducing RNA expression of Orai3 and \stim{}, however, and is currently being used to generate stable cell lines to further this study. Given the importance of SOCE to the adaptive immune response and related disease states, this initial work and future studies based upon it, have the potential to be important vehicles for contributing to the body of knowledge in the field of \Ca{} metabolism. Studies of diseases resulting from disrupted \Ca{} homeostasis, including forms of cancer and autoimmune diseases, may also benefit from the creation of this expression system.




%\newpage 

%\section{Discussion \& Future Work}
%It is possible that the endogenous \dorai{} and \dstim{} proteins will, on the basis of their similarity with the human orthologs, have some contribution to the \Ca{} measurements obtained. \\droso{} are very amenable to RNA interference (RNAi), which can be used to knock down specific genes, based on their sequence. Future work will be to create stable cell lines,  expressing our genes of interest. The vectors used, were designed with this goal in mind, and only Hygromycin B selection of S2 cells transfected with our construct remains. 
%Furthermore, by knocking down \dorai{} and \dstim{} using RNAi, and repeating the experiments, the contribution of the endogenous genes can be determined.

%More work is necessary to make this system useable for the purpose of drug discovery, but the foundation of this work has already been laid. Creation of stable cell lines expressing our genes of interest is currently being undertaken. These stably expressed \stim{} and Orai3 lines will allow us to answer questions on protein expression of the channel. Due to the unexpected result we get from \stim{} expression with 200 $\mu$ 2-APB, it is very unlikely that \stim{} is not being expressed. 
%The stable cell line will allow us to study the novel behavior of \stim{} in this model system. important questions such as, the channel heterologous \stim{} seems to be affecting, will be answered more easily due to the creation of this cell line. 
%************ Given that \stim{} expresses, it seems unlikely that Orai3, which was made in a similar manner is not expressing. 
%************ More likely it is interacting with \dorai{} and the result of these heteromeric channels are what we are looking at.

%See supplemental figure. GFP only, Stim1 + GFP transient transfection, STIM1 + GFP after 1 month of selection.
%See supplemental figure. No 2-APB in untransfected cells.  

%%%%%%%%%%%%%% NOTICE MEEEE %%%%%%%%%%%%%%%%%
%See Goto2010 for example of ORAI3 activation dependent on STIM. Precedent exists. 

 